\chapter{The algorithm}

\section{Description}

\section{Implementation}

% pseudocode
\begin{figure}[ht]
\caption{Algorithm for calculating the attack rules on mvPDAs}
\begin{algorithmic}[1]
\Function{AttackRules}{$mvPDA = (\Dmay, \Dmust)$}
  \State $rules ← ∅$
  \For{$P,Q,S,T ∈ Const(mvPDA), a ∈ Act(mvPDA), type ∈ \{\text{may},\text{must}\}$}
    \If{$type = \text{may}$}
      \State $lhs ← (P⋅S, Q⋅T)$
      \Comment{Attack from left-hand side for may rules}
    \Else
      \State $lhs ← (Q⋅S, P⋅Y)$
      \Comment{Attack from right-hand side for must rules}
    \EndIf
    %\State $lhs ← (p, q)(X, Y)$
    \For{$(P⋅S, a, p') ∈ Δ_{type}$}
      \State $rhs ← ∅$
      \For{$(Q⋅T, a, q') ∈ Δ_{type}$}
        \If{$type = \text{may}$}
          \State $newRhs ← (p', q')$
        \Else
          \State $newRhs ← (q', p')$
        \EndIf
        \State $rhs ← rhs ∪ \{ newRhs \}$
      \EndFor
      \State $rules ← rules ∪ \{(lhs, rhs)\}$
    \EndFor
  \EndFor
  \State \textbf{return} $rules$
\EndFunction
\end{algorithmic}
\end{figure}

\begin{figure}[ht]
\caption{Algorithm for combining attack rules}
\begin{algorithmic}[1]
  \Function{Combine}{$lhsRule = (lhs, lhsRhsSet), rhsRule = (rhsLhs, rhsSet)$}
  \State $rules ← ∅$
  \If{$\forall rhs ∈ rhsSet : size(rhs) ≤ 1$}
    \For{$lhsRhs ∈ lhsRhsSet : lhsRhs = rhsLhs⋅p$}
      \State $newRhs ← (lhsRhsSet ∖ lhsRhs) ∪ \{ rhs⋅p \mid rhs ∈ rhsSet \}$
      \State $rules ← rules ∪ \{ (lhs, newRhs) \}$
    \EndFor
  \EndIf
  \State \textbf{return} $rules$
\EndFunction
\end{algorithmic}
\end{figure}

\begin{figure}[ht]
\caption{Refinement algorithm for mvPDAs}
\begin{algorithmic}[1]
\Function{VPDARefinement}{$P⋅S, Q⋅T, mvPDA$}
  \Comment{$P⋅S ≤_m Q⋅T$ given $mvPDA$}
  \State $initial ← [P⋅S, Q⋅T]$
  \State $rules ← \Call{AttackRules}{mvPDA}$
  \While{$\exists lhsRule, rhsRule ∈ rules :
      \Call{Combine}{lhsRule, rhsRule} ⊄ rules $}
    \State $rules ← rules ∪ \Call{Combine}{lhsRule, rhsRule}$
  \EndWhile
  \State \textbf{return} $(initial, ∅) ∈ rules$
\EndFunction
\end{algorithmic}
\end{figure}

\section{Soundness and completeness}

Follows from theorem \label{theorem:refinement-tree} and
theorem \label{theorem:tree-attack} and

% proof: no refinement => algorithm outputs false
% proof: algorithm outputs false => no refinement
% termination proof
% termination proof

\section{Runtime}

% tests with complete number

\section{Optimizations}

% only keep smallest rhs rules

% exploration of reachable states

% hash map for looking up lhs/rhs

% early stopping

\section{Input and output}

% now command line
% future integration with gui
% similiar input file to existing programs

\section{Performance evaluation}

\section{Example}

Figure \ref{mvpda-p} and \ref{mvpda-q} define two mvPDA. The corresponding may transitions for the must transitions are implied.
The problem is to decide whether $p⋅S ≤_m q⋅S$.

\begin{figure*}[ht]
  \centering
  \begin{minipage}[b]{.4\textwidth}
    \begin{align*}
      P⋅S &\must[coin] P⋅M⋅S \\
      P⋅M &\must[coin] P⋅M⋅M \\
      P⋅M &\must[tea] T \\
      P⋅M &\must[coffee] c \\
      T⋅M &\must[tea] T \\
      T⋅S &\must[coin] P⋅M⋅S \\
      c⋅M &\must[coffee] c \\
      c⋅S &\must[coin] P⋅M⋅S
    \end{align*}
    \caption{mvPDA for process $P⋅S$}\label{mvpda-p}
  \end{minipage}\qquad
  \begin{minipage}[b]{.4\textwidth}
    \begin{align*}
      Q⋅S &\may[coin] Q⋅T⋅S \\
      Q⋅S &\may[coin] Q⋅C⋅S \\
      Q⋅T &\may[coin] Q⋅T⋅T \\
      Q⋅C &\may[coin] Q⋅C⋅C \\
      Q⋅T &\must[tea] Q \\
      Q⋅T &\may[coffee] Q \\
      Q⋅C &\may[tea] Q \\
      Q⋅C &\must[coffee] Q
    \end{align*}
    \caption{mvPDA for process $Q⋅S$}\label{mvpda-q}
  \end{minipage}
\end{figure*}

\begin{figure*}[ht]
  \centering
\begin{tikzpicture}[
  level 1/.style={level distance=6cm, sibling distance=3.5cm},
  level 2/.style={level distance=8cm, sibling distance=2cm},
  level 3/.style={level distance=7cm, sibling distance=2cm},
  edge from parent path={(\tikzparentnode.east) -- (\tikzchildnode.west)},
  scale=0.6,
  every node/.style={scale=0.6}
]
\node[state] {$(p⋅S,q⋅S)$}
    child {
        node[state] {$(p⋅M⋅S,q⋅T⋅S)$}
        child {
          node[state] {$(p⋅M⋅M⋅S,q⋅T⋅T⋅S)$}
          child {
            node[state] {$(c⋅M⋅T,q⋅T⋅S)$}
            node[attack] {$q⋅T \must[tea] q $}
            edge from parent
            node[defense] {$q⋅T \may[coffee] q $}
          }
          node[attack] {$p⋅M \may[coffee] c $}
          edge from parent
          node[defense] {$q⋅T \may[coin] q⋅T⋅T $}
        }
        node[attack] {$p⋅M \may[coin] p⋅M⋅M $}
        edge from parent
        node[defense] {$q⋅S \may[coin] q⋅T⋅S $}
    }
    child {
        node[state] {$(p⋅M⋅S,q⋅C⋅S)$}
        child {
          node[state] {$(p⋅M⋅M⋅S,q⋅C⋅C⋅S)$}
          child {
            node[state] {$(T⋅M⋅S,q⋅C⋅S)$}
            node[attack] {$q⋅C \must[coffee] q $}
            edge from parent
            node[defense] {$q⋅C \may[tea] q $}
          }
          node[attack] {$p⋅M \may[tea] T $}
          edge from parent
          node[defense] {$q⋅C \may[coin] q⋅C⋅C $}
        }
        node[attack] {$p⋅M \may[coin] p⋅M⋅M $}
        edge from parent
        node[defense] {$q⋅S \may[coin] q⋅C⋅S $}
    }
    node[attack] {$p⋅S \may[coin] p⋅M⋅S $}
    ;
\end{tikzpicture}
  \caption{Tree for winning strategy}\label{pq-tree}
\end{figure*}

\begin{figure*}[ht]
  \centering
\begin{tikzpicture}[
  level 1/.style={level distance=2cm, sibling distance=4cm},
  level 2/.style={level distance=8cm, sibling distance=2cm},
  level 3/.style={level distance=6cm, sibling distance=2cm},
  northchild/.style={edge from parent path={(\tikzparentnode.north) -- (\tikzchildnode.south)}},
  southchild/.style={edge from parent path={(\tikzparentnode.south) -- (\tikzchildnode.north)}},
  rightchild/.style={edge from parent path={(\tikzparentnode.east) -- (\tikzchildnode.west)}},
  scale=0.6,
  every node/.style={scale=0.6}
]
\node[state] {$(p⋅S,q⋅S) \attack[] \{ (p⋅M⋅S,q⋅C⋅S), (p⋅M⋅S,q⋅T⋅S) \}$}
    child[southchild] {
      node[state] {$(p⋅M,q⋅T) \attack[] \{ (p⋅M⋅M,q⋅T⋅T) \}$}
        child[rightchild] {
          node[state] {$(p⋅M,q⋅T) \attack[] \{ (c,q) \} $}
          child[rightchild] {
            node[state] {$(c⋅M,q⋅T) \attack[] ∅$}
            edge from parent
            node[defense] {return}
          }
          edge from parent
          node[defense] {call}
        }
        edge from parent
        node[defense] {call}
    }
    child[northchild] {
      node[state] {$(p⋅M,q⋅C) \attack[] \{ (p⋅M⋅M,q⋅C⋅C) \}$}
        child[rightchild] {
          node[state] {$(p⋅M,q⋅C) \attack[] \{ (T,q) \} $}
          child[rightchild] {
            node[state] {$(T⋅M,q⋅C) \attack[] ∅$}
            edge from parent
            node[defense] {return}
          }
          edge from parent
          node[defense] {call}
        }
        edge from parent
        node[defense] {call}
    }
    ;
\end{tikzpicture}
  \caption{Tree for winning strategy with attack rules}\label{pq-attack-tree}
\end{figure*}

\begin{figure*}[ht]
  \centering
\begin{tikzpicture}[
  level 1/.style={level distance=10cm, sibling distance=4cm},
  level 2/.style={level distance=8cm, sibling distance=2cm},
  state/.style={text centered, rectangle, draw},
  edge from parent path={(\tikzparentnode.east) -- (\tikzchildnode.west)},
  scale=0.6,
  every node/.style={scale=0.6}
]
\node[state] {$(p⋅S,q⋅S) \attack[] \{ (p⋅M⋅S,q⋅C⋅S), (p⋅M⋅S,q⋅T⋅S) \}$}
    child {
      node[state] {$(p⋅M,q⋅T) \attack[] \{ (c⋅M,q⋅T) \}$}
        child {
          node[state] {$(c⋅M,q⋅T) \attack[] ∅$}
          edge from parent
          node[defense] {internal}
        }
        edge from parent
        node[defense] {call}
    }
    child {
      node[state] {$(p⋅M,q⋅C) \attack[] \{ (T⋅M,q⋅C) \}$}
        child {
          node[state] {$(T⋅M,q⋅C) \attack[] ∅$}
          edge from parent
          node[defense] {internal}
        }
        edge from parent
        node[defense] {call}
    }
    ;
\end{tikzpicture}
  \caption{Merged tree for winning strategy with attack rules after one step}\label{pq-attack-tree-1}
\end{figure*}


\begin{figure*}[ht]
  \centering
\begin{tikzpicture}[
  level 1/.style={level distance=12cm, sibling distance=4cm},
  edge from parent path={(\tikzparentnode.east) -- (\tikzchildnode.west)},
  scale=0.6,
  every node/.style={scale=0.6}
]
\node[state] {$(p⋅S,q⋅S) \attack[] \{ (p⋅M⋅S,q⋅C⋅S), (p⋅M⋅S,q⋅T⋅S) \}$}
    child {
      node[state] {$(p⋅M,q⋅T) \attack[] ∅$}
        edge from parent
        node[defense] {call}
    }
    child {
      node[state] {$(p⋅M,q⋅C) \attack[] ∅$}
        edge from parent
        node[defense] {call}
    }
    ;
\end{tikzpicture}
  \caption{Merged tree for winning strategy with attack rules after two steps}\label{pq-attack-tree-2}
\end{figure*}


\begin{figure*}[ht]
  \centering
\begin{tikzpicture}[
  edge from parent path={(\tikzparentnode.east) -- (\tikzchildnode.west)},
  scale=0.6,
  every node/.style={scale=0.6}
]
\node[state] {$(p⋅S,q⋅S) \attack[] ∅$};
\end{tikzpicture}
  \caption{Final merged tree for winning strategy}\label{pq-attack-tree-3}
\end{figure*}
