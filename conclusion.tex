\chapter{Conclusion}

\section{Main results}

In this thesis, an algorithm for deciding refinement
between modal visibly pushdown automata was presented.
This was done by use of modal process rewrite systems
and a new representation of refinement by attack trees.

The theoretical part introduced the concepts and showed
correctness of the construction.
The practical part described the implementation,
including some optimizations, and analyzed the
complexity with theoretical bounds and a performance evaluation.
This showed the unavoidable exponential runtime and
conditions on input which produce or avoid it.

This will hopefully contribute to the usage of mvPDAs
in the fields applying modal transitions systems, such as in
system specifications and model checking, by providing
a sound and complete way to test refinement.

\section{Further extensions}

Even with the exponential runtime, there are certainly
more possible optimizations to improve the runtime.
One possibility would be deciding positive refinement in certain cases before
exploring the whole rule space, for example by showing that there are
loops from each attack transition.
It should also be possible to show that the worst-time complexity
is always $2^{\mathcal O(n^2)}$ instead of $2^{\mathcal O(n^6)}$.

Further development in this area would then developing algorithms to
decide refinement between other classes of mPRS where it is possible, for example
between finite mFSM and general mPRS \cite{BenesK12}. Also there could be
special treatment for mvBPA, for which the refinement is PTIME-complete.

The tool should then be integrated into the MoTraS system to be made
available for end users. Then more options could be added, for example
more verbose output such as optionally providing a counterexample
in case of negative refinement.

