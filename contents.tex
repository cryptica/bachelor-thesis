\begingroup

\tableofcontents

\listoffigures

%\listoftables % TODO remove if no tables

\let\cleardoublepage\relax
\let\clearpage\relax

\lstlistoflistings % TODO convert all listings to figures?

\endgroup
