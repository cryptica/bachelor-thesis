% TUM logo
\def\oTUM#1{%
\dimen1=#1\dimen1=.1143\dimen1
\dimen2=#1\dimen2=.419\dimen2
\dimen3=#1\dimen3=.0857\dimen3
\dimen0=#1\dimen0=.018\dimen0
\dimen4=\dimen1\advance\dimen4 by-\dimen0
\setbox1=\vbox{\hrule width\dimen0 height\dimen4 depth0pt}
\advance\dimen4 by\dimen2
\setbox8=\vbox{\hrule width\dimen0 height\dimen4 depth0pt}
\advance\dimen4 by-\dimen2\advance\dimen4 by-\dimen0
\setbox4=\vbox{\hrule width\dimen4 height\dimen0 depth0pt}
\advance\dimen4 by\dimen1\advance\dimen4 by\dimen3
\setbox6=\vbox{\hrule width\dimen4 height\dimen0 depth0pt}
\advance\dimen4 by\dimen3\advance\dimen4 by\dimen0
\setbox9=\vbox{\hrule width\dimen4 height\dimen0 depth0pt}
\advance\dimen4 by\dimen1
\setbox7=\vbox{\hrule width\dimen4 height\dimen0 depth0pt}
\dimen4=\dimen3
\setbox5=\vbox{\hrule width\dimen4 height\dimen0 depth0pt}
\advance\dimen4 by-\dimen0
\setbox2=\vbox{\hrule width\dimen4 height\dimen0 depth0pt}
\dimen4=\dimen2\advance\dimen4 by\dimen0
\setbox3=\vbox{\hrule width\dimen0 height\dimen4 depth0pt}
\setbox0=\vbox{\hbox{\box9\lower\dimen2\copy3\lower\dimen2\copy5
\lower\dimen2\copy3\box7}\kern-\dimen2\nointerlineskip
\hbox{\raise\dimen2\box1\raise\dimen2\box2\copy3\copy4\copy3
\raise\dimen2\copy5\copy3\box6\copy3\raise\dimen2\copy5\copy3\copy4\copy3
\raise\dimen2\box5\box3\box4\box8}}
\leavevmode\box0}

% math commands
\newcommand{\Dmay}{Δ_\text{may}}
\newcommand{\Dmust}{Δ_\text{must}}
\newcommand{\may}[1][]{\ifthenelse{\isempty{#1}{}}
  {\dashrightarrow}
  {\stackrel{\text{#1}}{\dashrightarrow}}
}
\newcommand{\must}[1][]{\ifthenelse{\isempty{#1}}
  {\longrightarrow}
  {\stackrel{\text{#1}}{\longrightarrow}}
}
\newcommand{\attack}[1][]{\ifthenelse{\isempty{#1}}
  {\longrightarrow_a}
  {\stackrel{\text{#1}}{\longrightarrow}_a}
}
\newcommand{\call}{\text{call}}
\newcommand{\return}{\text{return}}
\newcommand{\internal}{\text{internal}}
\newcommand{\atree}{\text{ atree}}
\newcommand{\leaf}{\text{leaf}}
\newcommand{\mc}{\mathcal}

% define theorems
\theoremstyle{definition}
\newtheorem{definition}{Definition}
\theoremstyle{plain}
\newtheorem{lemma}{Lemma}
\newtheorem{theorem}{Theorem}
\newtheorem{corollary}{Corollary}
\newtheorem{example}{Example}
\theoremstyle{remark}
\newtheorem{remark}{Remark}


% tikz options
\tikzset{  
  state/.style={text centered, rectangle, draw},
  openstate/.style={text centered, rectangle, rounded corners=3mm, draw},
  attack/.style={below=-0.7em},
  label/.style={left=5.0em},
  attlabel/.style={left=14.0em},
  defense/.style={above=-0.2em},
  grow=right,
  edge from parent/.style={->,draw,shorten >=5pt, shorten <=5pt},
}

% listings options
\lstset{
%  breakindent=0em,
%  language=XML,
%  basicstyle=\footnotesize,
%  numbers=left,
%  numberstyle=\footnotesize,
%  stepnumber=2,
%  numbersep=5pt,
%  backgroundcolor=\color{white},
%  showspaces=false,
%  showstringspaces=false,
%  showtabs=false,
%  frame=single,
%  tabsize=2,
%  captionpos=b,
  breaklines=true,
  breakatwhitespace=true,
  breakautoindent=true,
}
