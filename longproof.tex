
We want to show:
\begin{theorem}
  For an mvPDA $(\Dmay, \Dmust)$ with its mPRS attack relation $\mc B$, its
  induced mPRS induced attack relation $\mc C$,
  its induced MTS $(\mc P, \may[], \must[])$ and the attack relation $\mc A$ of
  the MTS:
  \[
    \mc A = \mc C
  \]
\end{theorem}

\begin{proof}
    \Rightarrow: Assume $r := ((p,q), S) ∈ \mc A$.
    We proof that $r ∈ \mc A$ by induction on the size of the inference tree of $r$ from
    a rule in the mPRS.
    \begin{enumerate}
      \item It was created by
        \begin{mathpar}
          \inferrule{
            \inferrule{(p, a, p') ∈ \Dmay}{p \may[a] p'} \, (1) 
          }
          {((p,q), \{ (p', q') \mid q \may[a] q' \}) ∈ \mc A} \, (1) 
        \end{mathpar}
        And as $|p| = |q| = 2$, each we have for each $q \may[a] q'$:
        \begin{mathpar}
          \inferrule{(q, a, q') ∈ \Dmay}{q \may[a] q'} \, (1) 
        \end{mathpar}
        Then we can infer
        \begin{mathpar}
          \inferrule{
            \inferrule{(p, a, p') ∈ \Dmay}
            {((p,q), \{ (p', q') \mid (q, a, q') ∈ \Dmay \}) ∈ \mc B} \, (1) 
          }
          { ((p, q), \{ (p', q') \mid q \may[a] q' \}) ∈ \mc C} \, (1) 
        \end{mathpar}
      \item It was created by
        \begin{mathpar}
          \inferrule{
            \inferrule{(q, a, q') ∈ \Dmust}{q \must[a] q'} \, (2) 
          }
          {((p,q), \{ (p', q') \mid p \must[a] p' \}) ∈ \mc A} \, (2) 
        \end{mathpar}
        Then we can infer
        \begin{mathpar}
          \inferrule{
            \inferrule{(q, a, q') ∈ \Dmust}
            {((p,q), \{ (p', q') \mid (p, a, p') ∈ \Dmust \}) ∈ \mc B} \, (2) 
          }
          { ((p, q), \{ (p', q') \mid p \must[a] p' \}) ∈ \mc C} \, (2) 
        \end{mathpar}
    \end{enumerate}
\end{proof}

\begin{definition}[mPRS induced attack relation]
  Let $\mc B$ be the mPRS attack relation of an mPRS.
  Then the induced attack relation $\mc C$ is
  The \emph{mPRS attack relation} over the set of states $\mc P^2 = \mc P × \mc P$ is
  is the relation $\mc C ⊆ \mc P × 2^{\mc P}$ given by the inference rules
  \begin{mathpar}
    \inferrule{((p, q), S) ∈ \mc B}{ ((p, q), S) ∈ \mc C }
      \, (1) \\
    \inferrule{((p, q), S) ∈ \mc C}{ (p⋅r, q⋅r), \{ (p'⋅r, q'⋅r) \mid (p',q') ∈ S \} ∈ \mc C }
      \, (S) \\
    \inferrule{((p, q), S) ∈ \mc C}{ (p\|r, q\|r), \{ (p'\|r, q'\|r) \mid (p',q') ∈ S \} ∈ \mc C}
      \, (P) \\
  \end{mathpar}
\end{definition}

As the inference chain is linear,
for any element $c = ( (p',q'),S' ) ∈ \mc C$, we can assign a generating element
$b = ( (p,q), S ) ∈ \mc B$. The size of the set is preserved with each rule, so
$|S| = |S'|$ and the left-hand side can only increase in size, so
$|p| ≤ |p'$ and $|q| ≤ |q'|$.
For any purely sequential process we have
$p \text{ prefix of } p'$ and
$q \text{ prefix of } q'$ and
Obviously $\mc B ⊆ \mc C$. % TODO


\begin{theorem}
  For any derivation tree for $( (P⋅S, q⋅T), ∅ )$
  constructable out of rules from $\mc A_v$,
  the algorithm eventually adds $( (P⋅S, q⋅T), ∅ )$ to $\mc A_v$.
\end{theorem}

\begin{proof}
  Let $T$ be a tree for $( (P⋅S, q⋅T), ∅ )$ with all components
  in $\mc A_v$. Proof that eventually $( (P⋅S, q⋅T), ∅ )$ is added
  to $\mc A_v$ by induction over the size of $T$:
  \begin{enumerate}
    \item $|T| = 1$: Then the tree just consists of the rule
      $( (P⋅S, q⋅T), ∅ ) ∈ \mc V$ which has already been added to $\mc A$.
    \item $|T| > 1$: Then there exists a non-empty path of the form
      $(\call + \internal)(\call + \internal + \return)^*$ starting at
      the root $(P⋅S, q⋅T)$.
      There can be no path of the form $(\return)(\call + \internal + \return)^*)$, as
      the first return rule would result in the state $(p,q)$ from which no further rules
      are appliable.

      Especially there exists a path of the form
      $(P⋅S,q⋅T) \attack[]^* (p'⋅S', q'⋅T') \attack[\call + \internal] (p''⋅S''⋅α, q''⋅T''⋅β)$
        such that all edges from $(p''⋅S'', q''⋅T'')$ are return transitions.
        This state can be found by following all call or internal transitions from the root.
        Note that may be no transitions from the last state node if it is a leaf node.

        Let $( (p'⋅S',q'⋅T'), \{(p''⋅S''⋅α, q''⋅T''⋅β)\} ∪ S'\} ∈ \mc A_v$ be the rule
        leading to that state. As it is a call or internal rule, we have
        $0 ≤ |α| ≤ 1$ and $0 ≤ |β| ≤ 1$
        Also let $( (p''⋅S'',q''⋅T''), S'') ∈ \mc A_v$ be the rule leading from that state.
        As they are all return transitions, all elements in $S''$ have the form $(p''',q''')$.

        Then we can add a new rule to $\mc A_v$ by combining the two rules into
        $(p'⋅S',q'⋅T', S' ∪ \{ (p'''⋅α,q'''⋅β) ∈ S'' \})$. This is a valid rule as
        $|p'''⋅α| ≤ 2$ and $|q'''⋅β| ≤ 2$. With this rule, we have a derivation for
        $( (P⋅S, q⋅T), ∅ )$ with a tree $T'$ in which these two rules are replaced by
        the new rule. So we have $|T'| = |T| - 1$ and by induction hypothesis the
        algorithm eventually adds $( (P⋅S, q⋅T), ∅ )$ to $\mc A_V$ 

  \end{enumerate} 
  and $T$ is a derivation tree of
  $( (P⋅S, q⋅T), ∅ )$ %in $A$, then 
\end{proof}







  By induction on $|open(T)|$
  \begin{enumerate}
    \item $|open(T)| = 1$:
      If $(p',q') ∈ O$ 
      we can create $S = ((p,q), O ∖ \{(p', q')\}, R) \atree$
      with $open(S) = open(T) ∖ \{(p', q')\} ∪ open(R)$
      Otherwise $(p',q') ∈ T'$ for $T' ∈ C$

      T = ((p,q), O, ∅)$:
    \item $T = ((p,q), O, C ∪ T')$:
      Then $T$ was created from
      $T'' = ((p,q), O' \uplus \{(p'',q'')\}, C) \atree$ and
      $T' = ((p'',q''), ␣, ␣) \atree$.

      If $(p',q') ∈ O$
      As $open(T) = open(T'') ∖ \{(p'',q'')\} ∪ open(T')$,
      we could have the cases
      \begin{enumerate}
        \item $(p',q') = (p'',q'')$:
        \item $(p',q') ∈ open(T')$:
          With the induction hypothesis
          from $T'$ and $R$ we get a tree
          $S' = ((p'',q''), ␣, ␣) \atree$ with
          $open(S') = (open(T') ∖ \{ (p',q') \} ) ∪ open(R)$
          Then we can combine $T''$ and $S'$ to
          $S = ((p,q), O', C ∪ S') \atree$ with
          $open(S) = O' ∪ (⋃_{T'∈C}open(T')) ∪ open(S') =
                     O' ∪ (⋃_{T'∈C}open(C)) ∪ ((open(T') ∖ \{ (p',q') \} ) ∪ open(R)) = 
                     (O' ∪ (⋃_{T'∈C}open(C)) ∪ (open(T')) ∖ \{ (p',q') \} ) ∪ open(R) = 
                     (open(T) ∖ \{ (p',q') \} ) ∪ open(T'))$
        \item $(p',q') ∉ open(T')$: Then $(p',q') ∈ open(T'') ∖ \{(p'',q'')\}$.
          By induction hypothesis
          from $T''$ and $T'$ we get a tree
          $S' = ((p'',q''), O''', C''') \atree$ with
          $open(S') = (open(T'') ∖ \{ (p',q') \} ) ∪ open(T')$.
          Then we can combine $T'''$ and $S'$ to
          $S = ((p,q), O', C ∪ S') \atree$ with
          $open(S) = O' ∪ (⋃_{T'∈C}open(C)) ∪ open(S') =
                     O' ∪ (⋃_{T'∈C}open(C)) ∪ ((open(T'') ∖ \{ (p',q') \} ) ∪ open(T')) = 
                     (O' ∪ (⋃_{T'∈C}open(C)) ∪ (open(T'')) ∖ \{ (p',q') \} ) ∪ open(T') = 
                     (open(T) ∖ \{ (p',q') \} ) ∪ open(T'))$
        \item $(p',q') ∈ open(T''')$
      \end{enumerate}
  \end{enumerate}







\begin{lemma}
  \label{lemma:attack-to-tree}
  If $(p,q) \attack S$, then there is a tree
  $T$ with $root(T) = (p,q)$ and $open(T) = S$.
\end{lemma}
\begin{proof}
  By induction on the construction of $(p,q) \attack S$:
  \begin{enumerate}
    \item It was constructed by rule 1 from $(p, a, p') ∈ \Dmay$. Then there is
      an attacking transition $p \may[a] p'$ and 
      for every $(q, a, q') ∈ \Dmay$ there is an induced defending transition
      $q \may[a] q'$.
      Then $S = \{ (p',q') | q \may[a] q' \}$ and by attack tree inference rule 1
      there is $T = ((p, q), S, ∅)$ with $open(T) = S$.
    \item It was constructed by rule 2 from $(q, a, q') ∈ \Dmust$. Then there is
      an attacking transition $q \must[a] q'$ and 
      for every $(p, a, p') ∈ \Dmay$ there is an induced defending transition
      $p \must[a] p'$.
      Then $S = \{ (p',q') | p \may[a] p' \}$ and by attack tree inference rule 2
      there is $T = ((p, q), S, ∅)$ with $open(T) = S$.
    \item It was constructed by rule 3 from
      $(p,q) \attack \{(p'⋅P,q'⋅Q)\} \uplus S''$ and
      $(p',q') \attack S'$ with $S = S'' ∪ S'''$ and
      $S''' = \{  (p''⋅P, q''⋅Q) \mid (p'',q'') ∈ S' \}$.
      Then by induction hypothesis there is
      a tree $T'$ with $root(T') = (p',q')$ and $open(T') = S'$ and
      a tree $T''$ with $root(T'') = (p,q)$ and $open(T'') = S'' \uplus \{(p'⋅P,q'⋅Q)\}$.
      By appying lemma \ref{lemma:tree-lift} on $T'$ there is a tree
      $T'''$ with $root(T''') = (p'⋅P,q'⋅Q)$,
      $open(T''') = O''' \uplus \{(p'⋅P,q'⋅Q)\}$ and
      $O''' = \{ (p''⋅P, q''⋅Q) \mid (p'',q'') ∈ S' \} = S'''$.
      By applying lemma \ref{lemma:tree-composition} on $T''$ and $T'''$ there is
      a tree $T$ with $root(T) = (p,q)$ and $open(T) = S'' ∪ S''' = S$.
    \item It was constructed by rule 4 from
      $(p,q) \attack S'' \uplus \{(p',q')\} $ and
      $(p',q') \attack S'$ with $S = S'' ∪ S'$.
      Then by induction hypothesis there is
      a tree $T'$ with $root(T') = (p',q')$ and $open(T') = S'$ and
      a tree $T''$ with $root(T'') = (p,q)$ and $open(T'') = S'' \uplus \{(p',q')\}$.
      By applying lemma \ref{lemma:tree-composition} on $T'$ and $T''$ there is
      a tree $T$ with $root(T) = (p,q)$ with $open(T) = S'' ∪ S' = S$.
  \end{enumerate}
\end{proof}

